% ex: tw=74 
% ex: tw=74 ts=2 sw=2 noet sts=2
\documentclass[10pt]{article}
\usepackage[utf8]{inputenc}
\usepackage{amsmath}
\usepackage{xkeyval}
\usepackage[bookmarksnumbered,frenchlinks]{hyperref}
%\hypersetup{pdfborder=0 0 0}
\usepackage{multirow}
\usepackage[english]{babel}
\usepackage{fullpage}
\usepackage{tabulary}
\usepackage{tabularx}
%\usepackage{natbib}
\usepackage[all]{hypcap}
\usepackage{hyperref}
\usepackage{framed}
\usepackage{fullpage}
\usepackage{graphicx}
\usepackage{listings}
\usepackage{subfig}
\usepackage{verbatim}
\usepackage{float}
\usepackage{datatool}
%\floatstyle{boxed} 
%\restylefloat{figure}

\title{\textbf{Homework 3}
Linux 2.4 Schedulers}

\author{Michael Koval and Cody Schafer}
\date{\today}
\begin{document}
\maketitle

\section{Project}

Implementing two distinct (though related) schedulers within the Linux
kernel, version 2.4, replacing the existing scheduling. One of the
schedulers was written to favor tasks with a low resident set size (RSS),
while the other favors tasks with a high RSS.

\section{Results}

Kernel binaries with the 'thin' (low rss favored), 'thick'  (high rss
favored), and 'normal' (the traditional Linux kernel scheduler) where
built. A modified version of the 'access pages' program was run on a
each of the kernels. 

\section{Implementation}

Each task has all of it's data contained within a respective
\texttt{task\_struct}. Within this structure is the memory map of the task
(the \texttt{mm} field) which keeps track of (among other things) the
resident set size (in \texttt{rss}) and the total virtual memory (in
\texttt{total\_vm}).

The original scheduler implementation has a function called
\texttt{goodness} which is given a task and the current memory map and
returns a value indicating how good it would be to schedule the given
task. Callers of this function treat negative values specially, and treat
zero specially in a distinct manner. They also make some assumptions about
the range of the return, typically expecting it to be addable to an
\texttt{int} without any overflow.

Replacing the existing scheduling with one using the resident set size
was accomplished by modifying \texttt{goodness} to return a value based on
a range clipped RSS (so that caller expectations are met to some degree).

It was initially expected that the above modification would be sufficient
for a functioning scheduler, however, on testing the kernel would hang
prior to userspace execution. To prevent this from occurring, the
\texttt{preemption\_goodness} function (which operates similarly to
\texttt{goodness} but compares two tasks and determines how good it would
be to switch to the new one) was modified to favor task switching in the
case where the goodness of the tasks matched. As all kernel tasks lack a
memory map (and thus are assigned the same goodness) and many tasks may
have the same number of pages in their resident set (and thus have
identical RSS's), exactly equal goodness values are extremely likely to
occur in these scheduler revisions.

\section{Further Possibilities}

Rescheduling of the tasks should occur each time a priority changes, thus
each time the RSS of a task changes. This would require the memory
management code to have knowledge of the task to which the memory map it
is modifying belongs all the way through, or would mean that the
accounting of rss would need to be centralized to a top level function.
The current structure of the memory management code is not conducive to
tight interaction with tasks, and thus would require large amounts of
modification to successfully implement lower latency scheduling dependent
on RSS. In the present implementation, rescheduling only occurs when a
task yields or preforms some blocking task (in the case of userspace
tasks, a syscall).

Setting the \texttt{need\_reschedule} flag could have been done in the mm
code (calling schedule here is probably not possible due to preemption
rules).


\end{document}
